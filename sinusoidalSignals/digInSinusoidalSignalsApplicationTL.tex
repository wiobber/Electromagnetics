\documentclass{ximera}  

%\usepackage{todonotes}
%\usepackage{mathtools} %% Required for wide table Curl and Greens
%\usepackage{cuted} %% Required for wide table Curl and Greens
\newcommand{\todo}{}
\usepackage{caption}
\usepackage{esint} % for \oiint
\ifxake%%https://math.meta.stackexchange.com/questions/9973/how-do-you-render-a-closed-surface-double-integral
\renewcommand{\oiint}{{\large\bigcirc}\kern-1.56em\iint}
\fi


\graphicspath{
  {./}
  {./jpg/}
  {../jpg/}
  {./ximeraTutorial/}
  {./basicPhilosophy/}
  {./functionsOfSeveralVariables/}
  {./normalVectors/}
  {./lagrangeMultipliers/}
  {./vectorFields/}
  {./greensTheorem/}
  {./shapeOfThingsToCome/}
  {./dotProducts/}
  {./partialDerivativesAndTheGradientVector/}
  {../productAndQuotientRules/exercises/}
  {../motionAndPathsInSpace/exercises/}
  {../normalVectors/exercisesParametricPlots/}
  {../continuityOfFunctionsOfSeveralVariables/exercises/}
  {../partialDerivativesAndTheGradientVector/exercises/}
  {../directionalDerivativeAndChainRule/exercises/}
  {../commonCoordinates/exercisesCylindricalCoordinates/}
  {../commonCoordinates/exercisesSphericalCoordinates/}
  {../greensTheorem/exercisesCurlAndLineIntegrals/}
  {../greensTheorem/exercisesDivergenceAndLineIntegrals/}
  {../shapeOfThingsToCome/exercisesDivergenceTheorem/}
  {../greensTheorem/}
  {../shapeOfThingsToCome/}
  {../separableDifferentialEquations/exercises/}
  {vectorFields/}
}

\newcommand{\mooculus}{\textsf{\textbf{MOOC}\textnormal{\textsf{ULUS}}}}

\usepackage{pgfplots}
% \pgfplotsset{compat=1.15}

\usepackage{tkz-euclide}\usepackage{tikz}
\usepackage{tikz-cd}
\usetikzlibrary{arrows}
\tikzset{>=stealth,commutative diagrams/.cd,
  arrow style=tikz,diagrams={>=stealth}} %% cool arrow head
\tikzset{shorten <>/.style={ shorten >=#1, shorten <=#1 } } %% allows shorter vectors

\usetikzlibrary{backgrounds} %% for boxes around graphs
\usetikzlibrary{shapes,positioning}  %% Clouds and stars
\usetikzlibrary{matrix} %% for matrix
\usepgfplotslibrary{polar} %% for polar plots
\usepgfplotslibrary{fillbetween} %% to shade area between curves in TikZ
% \usetkzobj{all}
\usepackage[makeroom]{cancel} %% for strike outs
%\usepackage{mathtools} %% for pretty underbrace % Breaks Ximera
%\usepackage{multicol}
\usepackage{pgffor} %% required for integral for loops



%% http://tex.stackexchange.com/questions/66490/drawing-a-tikz-arc-specifying-the-center
%% Draws beach ball
\tikzset{pics/carc/.style args={#1:#2:#3}{code={\draw[pic actions] (#1:#3) arc(#1:#2:#3);}}}



\usepackage{array}
\setlength{\extrarowheight}{+.1cm}
\newdimen\digitwidth
\settowidth\digitwidth{9}
\def\divrule#1#2{
\noalign{\moveright#1\digitwidth
\vbox{\hrule width#2\digitwidth}}}





\newcommand{\RR}{\mathbb R}
\newcommand{\R}{\mathbb R}
\newcommand{\N}{\mathbb N}
\newcommand{\Z}{\mathbb Z}

\newcommand{\sagemath}{\textsf{SageMath}}


%\renewcommand{\d}{\,d\!}
\renewcommand{\d}{\mathop{}\!d}
\newcommand{\dd}[2][]{\frac{\d #1}{\d #2}}
\newcommand{\pp}[2][]{\frac{\partial #1}{\partial #2}}
\renewcommand{\l}{\ell}
\newcommand{\ddx}{\frac{d}{\d x}}

\newcommand{\zeroOverZero}{\ensuremath{\boldsymbol{\tfrac{0}{0}}}}
\newcommand{\inftyOverInfty}{\ensuremath{\boldsymbol{\tfrac{\infty}{\infty}}}}
\newcommand{\zeroOverInfty}{\ensuremath{\boldsymbol{\tfrac{0}{\infty}}}}
\newcommand{\zeroTimesInfty}{\ensuremath{\small\boldsymbol{0\cdot \infty}}}
\newcommand{\inftyMinusInfty}{\ensuremath{\small\boldsymbol{\infty - \infty}}}
\newcommand{\oneToInfty}{\ensuremath{\boldsymbol{1^\infty}}}
\newcommand{\zeroToZero}{\ensuremath{\boldsymbol{0^0}}}
\newcommand{\inftyToZero}{\ensuremath{\boldsymbol{\infty^0}}}



\newcommand{\numOverZero}{\ensuremath{\boldsymbol{\tfrac{\#}{0}}}}
\newcommand{\dfn}{\textbf}
%\newcommand{\unit}{\,\mathrm}
\newcommand{\unit}{\mathop{}\!\mathrm}
\newcommand{\eval}[1]{\bigg[ #1 \bigg]}
\newcommand{\seq}[1]{\left( #1 \right)}
\renewcommand{\epsilon}{\varepsilon}
\renewcommand{\phi}{\varphi}


\renewcommand{\iff}{\Leftrightarrow}

\DeclareMathOperator{\arccot}{arccot}
\DeclareMathOperator{\arcsec}{arcsec}
\DeclareMathOperator{\arccsc}{arccsc}
\DeclareMathOperator{\si}{Si}
\DeclareMathOperator{\scal}{scal}
\DeclareMathOperator{\sign}{sign}


%% \newcommand{\tightoverset}[2]{% for arrow vec
%%   \mathop{#2}\limits^{\vbox to -.5ex{\kern-0.75ex\hbox{$#1$}\vss}}}
\newcommand{\arrowvec}[1]{{\overset{\rightharpoonup}{#1}}}
%\renewcommand{\vec}[1]{\arrowvec{\mathbf{#1}}}
\renewcommand{\vec}[1]{{\overset{\boldsymbol{\rightharpoonup}}{\mathbf{#1}}}\hspace{0in}}

\newcommand{\point}[1]{\left(#1\right)} %this allows \vector{ to be changed to \vector{ with a quick find and replace
\newcommand{\pt}[1]{\mathbf{#1}} %this allows \vec{ to be changed to \vec{ with a quick find and replace
\newcommand{\Lim}[2]{\lim_{\point{#1} \to \point{#2}}} %Bart, I changed this to point since I want to use it.  It runs through both of the exercise and exerciseE files in limits section, which is why it was in each document to start with.

\DeclareMathOperator{\proj}{\mathbf{proj}}
\newcommand{\veci}{{\boldsymbol{\hat{\imath}}}}
\newcommand{\vecj}{{\boldsymbol{\hat{\jmath}}}}
\newcommand{\veck}{{\boldsymbol{\hat{k}}}}
\newcommand{\vecl}{\vec{\boldsymbol{\l}}}
\newcommand{\uvec}[1]{\mathbf{\hat{#1}}}
\newcommand{\utan}{\mathbf{\hat{t}}}
\newcommand{\unormal}{\mathbf{\hat{n}}}
\newcommand{\ubinormal}{\mathbf{\hat{b}}}

\newcommand{\dotp}{\bullet}
\newcommand{\cross}{\boldsymbol\times}
\newcommand{\grad}{\boldsymbol\nabla}
\newcommand{\divergence}{\grad\dotp}
\newcommand{\curl}{\grad\cross}
%\DeclareMathOperator{\divergence}{divergence}
%\DeclareMathOperator{\curl}[1]{\grad\cross #1}
\newcommand{\lto}{\mathop{\longrightarrow\,}\limits}

\renewcommand{\bar}{\overline}

\colorlet{textColor}{black}
\colorlet{background}{white}
\colorlet{penColor}{blue!50!black} % Color of a curve in a plot
\colorlet{penColor2}{red!50!black}% Color of a curve in a plot
\colorlet{penColor3}{red!50!blue} % Color of a curve in a plot
\colorlet{penColor4}{green!50!black} % Color of a curve in a plot
\colorlet{penColor5}{orange!80!black} % Color of a curve in a plot
\colorlet{penColor6}{yellow!70!black} % Color of a curve in a plot
\colorlet{fill1}{penColor!20} % Color of fill in a plot
\colorlet{fill2}{penColor2!20} % Color of fill in a plot
\colorlet{fillp}{fill1} % Color of positive area
\colorlet{filln}{penColor2!20} % Color of negative area
\colorlet{fill3}{penColor3!20} % Fill
\colorlet{fill4}{penColor4!20} % Fill
\colorlet{fill5}{penColor5!20} % Fill
\colorlet{gridColor}{gray!50} % Color of grid in a plot

\newcommand{\surfaceColor}{violet}
\newcommand{\surfaceColorTwo}{redyellow}
\newcommand{\sliceColor}{greenyellow}




\pgfmathdeclarefunction{gauss}{2}{% gives gaussian
  \pgfmathparse{1/(#2*sqrt(2*pi))*exp(-((x-#1)^2)/(2*#2^2))}%
}


%%%%%%%%%%%%%
%% Vectors
%%%%%%%%%%%%%

%% Simple horiz vectors
\renewcommand{\vector}[1]{\left\langle #1\right\rangle}


%% %% Complex Horiz Vectors with angle brackets
%% \makeatletter
%% \renewcommand{\vector}[2][ , ]{\left\langle%
%%   \def\nextitem{\def\nextitem{#1}}%
%%   \@for \el:=#2\do{\nextitem\el}\right\rangle%
%% }
%% \makeatother

%% %% Vertical Vectors
%% \def\vector#1{\begin{bmatrix}\vecListA#1,,\end{bmatrix}}
%% \def\vecListA#1,{\if,#1,\else #1\cr \expandafter \vecListA \fi}

%%%%%%%%%%%%%
%% End of vectors
%%%%%%%%%%%%%

%\newcommand{\fullwidth}{}
%\newcommand{\normalwidth}{}



%% makes a snazzy t-chart for evaluating functions
%\newenvironment{tchart}{\rowcolors{2}{}{background!90!textColor}\array}{\endarray}

%%This is to help with formatting on future title pages.
\newenvironment{sectionOutcomes}{}{}



%% Flowchart stuff
%\tikzstyle{startstop} = [rectangle, rounded corners, minimum width=3cm, minimum height=1cm,text centered, draw=black]
%\tikzstyle{question} = [rectangle, minimum width=3cm, minimum height=1cm, text centered, draw=black]
%\tikzstyle{decision} = [trapezium, trapezium left angle=70, trapezium right angle=110, minimum width=3cm, minimum height=1cm, text centered, draw=black]
%\tikzstyle{question} = [rectangle, rounded corners, minimum width=3cm, minimum height=1cm,text centered, draw=black]
%\tikzstyle{process} = [rectangle, minimum width=3cm, minimum height=1cm, text centered, draw=black]
%\tikzstyle{decision} = [trapezium, trapezium left angle=70, trapezium right angle=110, minimum width=3cm, minimum height=1cm, text centered, draw=black]


\title{Engineering Design}  
\author{Milica Markovic}
\outcome{Derive equation of a wave on a transmission line by considering finite propagation speed of electric signals.Determine whether the transmission-line theory should be used based on line length and signal frequency.}
\begin{document}  
\begin{abstract}  
The purpose of this section is to show one application of sinusoidal signals to engineering design.
\end{abstract}  
\maketitle

When do we have to use the transmission-line theory? The answer to this question depends on the time-delay (or phase shift) that the line introduces. If the phase shift is small, for example, $\approx 2^0$, between the signal at the generator and the load, we don't have to use the transmission line theory, but if the phase between signals is much higher than  $\approx 2^0$, then we do have to use it.

We will now derive an expression for the phase shift on a transmission line between the generator and the load.

If we have a sinusoidal generator at one end of the transmission line, what is the signal at the other end?

The signal at the generator is 

\begin{eqnarray}
v_{AA^`}(t)=A cos(\omega t)
\end{eqnarray}

At the other end, the transmission line the signal is delayed.


\begin{eqnarray}
v_{BB^`}(t)=v_{AA^`}(t-T) \\
v_{BB^`}(t)=v_{AA^`}(t-\frac{l}{c}) \\
v_{BB^`}(t)=A cos(\omega (t - \frac{l}{c}))  \\
v_{BB^`}(t)=A cos(\omega t - \omega \frac{l}{c}) \\
v_{BB^`}(t)=A cos(\omega t -  \frac{\omega }{c} l) 
\end{eqnarray}

Since we know that angular frequency is  $\omega = 2 \pi f$


\begin{eqnarray}
v_{BB^`}(t)=A cos(\omega t -  \frac{ 2 \pi f }{c} l)
\end{eqnarray}

The quantity $\frac{c}{f}$ is called the wavelength $\lambda$, and it represents the distance between two maximums of a signal on a transmission line.  $\lambda$ is a period of the signal on a transmission line over distance, and the units are meters. 
It is analogous to the signal period $T=\frac{2 \pi}{\omega}$, but $\lambda$ is a period in space, not time.


\begin{eqnarray}
v_{BB^`}(t) = A \, cos(\omega t -  \frac{ 2 \pi }{\lambda} l) %\label{tllength1}
\end{eqnarray}

The quantity $ \frac{ 2 \pi }{\lambda} $ is called the propagation constant $\beta$, and it is analogous to the angular frequency $\omega$ of a signal, however, it represents how fast the signal is changing over distance and not time.


Finally, the expression for the voltage at BB end is


\begin{eqnarray}
v_{BB^`}(t)=A \, cos(\omega t - \beta l) \\
v_{BB^`}(t)=A \, cos(\omega t - \Psi)
\end{eqnarray}

We see that at BB' the signal will experience a phase shift.
We will later show that the solution of the wave equation is the equation above. We will derive the wave equation   from the Telegrapher's
equations.

Now let's see how the length of the line $l$ affects the voltage at the
end BB'. Look at Equation \ref{tllength1}.
The signal will experience a phase shift of $2\pi \frac{l}{\lambda}$. If this phase shift is small, there will not be much difference between
the signal's phase between the generator and the load. In this case, we don't have to use the transmission line theory to account for the line's effects.
If the phase shift is significant, then we do have to use the transmission line theory. Let's look at some numerical examples.

\begin{enumerate}
\item If $\frac{l}{\lambda} < 0.01$ then the angle $2 \pi
\frac{l}{\lambda}$ is of the order of 0.0314 rad or about 2$^0$. In this case, the
phase is obviosly something that we don't have to worry about. When
the length of the transmission line is much smaller than $\lambda$, $l<<\frac{\lambda}{100}$
the wave propagation on the line can be ingnored.
\item If  $\frac{l}{\lambda} > 0.01$, say  $\frac{l}{\lambda} =0.1$,
then the phase is 20$^0$, which is a significant phase shift. In this
case it may be necessary to account fro transmission line effects.
\end{enumerate}


\begin{question}  
Do we have to use transmission-line theory if the length of the line is 5000km, and the frequency of the signal is 60Hz? 
\begin{multipleChoice}  
\choice[correct]{Yes}  
\choice{No}    
\end{multipleChoice}  
\end{question} 



\begin{question}  
Do we have to use transmission-line theory if the length of the line is 2\,cm, and the frequency of the signal is 10\,GHz? 
\begin{multipleChoice}  
\choice[correct]{Yes}  
\choice{No}    
\end{multipleChoice}  
\end{question} 




\end{document}



                                                                   



                                       




                                    















