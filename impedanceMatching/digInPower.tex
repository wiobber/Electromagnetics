\documentclass{ximera}  


%\usepackage{todonotes}
%\usepackage{mathtools} %% Required for wide table Curl and Greens
%\usepackage{cuted} %% Required for wide table Curl and Greens
\newcommand{\todo}{}
\usepackage{caption}
\usepackage{esint} % for \oiint
\ifxake%%https://math.meta.stackexchange.com/questions/9973/how-do-you-render-a-closed-surface-double-integral
\renewcommand{\oiint}{{\large\bigcirc}\kern-1.56em\iint}
\fi


\graphicspath{
  {./}
  {./jpg/}
  {../jpg/}
  {./ximeraTutorial/}
  {./basicPhilosophy/}
  {./functionsOfSeveralVariables/}
  {./normalVectors/}
  {./lagrangeMultipliers/}
  {./vectorFields/}
  {./greensTheorem/}
  {./shapeOfThingsToCome/}
  {./dotProducts/}
  {./partialDerivativesAndTheGradientVector/}
  {../productAndQuotientRules/exercises/}
  {../motionAndPathsInSpace/exercises/}
  {../normalVectors/exercisesParametricPlots/}
  {../continuityOfFunctionsOfSeveralVariables/exercises/}
  {../partialDerivativesAndTheGradientVector/exercises/}
  {../directionalDerivativeAndChainRule/exercises/}
  {../commonCoordinates/exercisesCylindricalCoordinates/}
  {../commonCoordinates/exercisesSphericalCoordinates/}
  {../greensTheorem/exercisesCurlAndLineIntegrals/}
  {../greensTheorem/exercisesDivergenceAndLineIntegrals/}
  {../shapeOfThingsToCome/exercisesDivergenceTheorem/}
  {../greensTheorem/}
  {../shapeOfThingsToCome/}
  {../separableDifferentialEquations/exercises/}
  {vectorFields/}
}

\newcommand{\mooculus}{\textsf{\textbf{MOOC}\textnormal{\textsf{ULUS}}}}

\usepackage{pgfplots}
% \pgfplotsset{compat=1.15}

\usepackage{tkz-euclide}\usepackage{tikz}
\usepackage{tikz-cd}
\usetikzlibrary{arrows}
\tikzset{>=stealth,commutative diagrams/.cd,
  arrow style=tikz,diagrams={>=stealth}} %% cool arrow head
\tikzset{shorten <>/.style={ shorten >=#1, shorten <=#1 } } %% allows shorter vectors

\usetikzlibrary{backgrounds} %% for boxes around graphs
\usetikzlibrary{shapes,positioning}  %% Clouds and stars
\usetikzlibrary{matrix} %% for matrix
\usepgfplotslibrary{polar} %% for polar plots
\usepgfplotslibrary{fillbetween} %% to shade area between curves in TikZ
% \usetkzobj{all}
\usepackage[makeroom]{cancel} %% for strike outs
%\usepackage{mathtools} %% for pretty underbrace % Breaks Ximera
%\usepackage{multicol}
\usepackage{pgffor} %% required for integral for loops



%% http://tex.stackexchange.com/questions/66490/drawing-a-tikz-arc-specifying-the-center
%% Draws beach ball
\tikzset{pics/carc/.style args={#1:#2:#3}{code={\draw[pic actions] (#1:#3) arc(#1:#2:#3);}}}



\usepackage{array}
\setlength{\extrarowheight}{+.1cm}
\newdimen\digitwidth
\settowidth\digitwidth{9}
\def\divrule#1#2{
\noalign{\moveright#1\digitwidth
\vbox{\hrule width#2\digitwidth}}}





\newcommand{\RR}{\mathbb R}
\newcommand{\R}{\mathbb R}
\newcommand{\N}{\mathbb N}
\newcommand{\Z}{\mathbb Z}

\newcommand{\sagemath}{\textsf{SageMath}}


%\renewcommand{\d}{\,d\!}
\renewcommand{\d}{\mathop{}\!d}
\newcommand{\dd}[2][]{\frac{\d #1}{\d #2}}
\newcommand{\pp}[2][]{\frac{\partial #1}{\partial #2}}
\renewcommand{\l}{\ell}
\newcommand{\ddx}{\frac{d}{\d x}}

\newcommand{\zeroOverZero}{\ensuremath{\boldsymbol{\tfrac{0}{0}}}}
\newcommand{\inftyOverInfty}{\ensuremath{\boldsymbol{\tfrac{\infty}{\infty}}}}
\newcommand{\zeroOverInfty}{\ensuremath{\boldsymbol{\tfrac{0}{\infty}}}}
\newcommand{\zeroTimesInfty}{\ensuremath{\small\boldsymbol{0\cdot \infty}}}
\newcommand{\inftyMinusInfty}{\ensuremath{\small\boldsymbol{\infty - \infty}}}
\newcommand{\oneToInfty}{\ensuremath{\boldsymbol{1^\infty}}}
\newcommand{\zeroToZero}{\ensuremath{\boldsymbol{0^0}}}
\newcommand{\inftyToZero}{\ensuremath{\boldsymbol{\infty^0}}}



\newcommand{\numOverZero}{\ensuremath{\boldsymbol{\tfrac{\#}{0}}}}
\newcommand{\dfn}{\textbf}
%\newcommand{\unit}{\,\mathrm}
\newcommand{\unit}{\mathop{}\!\mathrm}
\newcommand{\eval}[1]{\bigg[ #1 \bigg]}
\newcommand{\seq}[1]{\left( #1 \right)}
\renewcommand{\epsilon}{\varepsilon}
\renewcommand{\phi}{\varphi}


\renewcommand{\iff}{\Leftrightarrow}

\DeclareMathOperator{\arccot}{arccot}
\DeclareMathOperator{\arcsec}{arcsec}
\DeclareMathOperator{\arccsc}{arccsc}
\DeclareMathOperator{\si}{Si}
\DeclareMathOperator{\scal}{scal}
\DeclareMathOperator{\sign}{sign}


%% \newcommand{\tightoverset}[2]{% for arrow vec
%%   \mathop{#2}\limits^{\vbox to -.5ex{\kern-0.75ex\hbox{$#1$}\vss}}}
\newcommand{\arrowvec}[1]{{\overset{\rightharpoonup}{#1}}}
%\renewcommand{\vec}[1]{\arrowvec{\mathbf{#1}}}
\renewcommand{\vec}[1]{{\overset{\boldsymbol{\rightharpoonup}}{\mathbf{#1}}}\hspace{0in}}

\newcommand{\point}[1]{\left(#1\right)} %this allows \vector{ to be changed to \vector{ with a quick find and replace
\newcommand{\pt}[1]{\mathbf{#1}} %this allows \vec{ to be changed to \vec{ with a quick find and replace
\newcommand{\Lim}[2]{\lim_{\point{#1} \to \point{#2}}} %Bart, I changed this to point since I want to use it.  It runs through both of the exercise and exerciseE files in limits section, which is why it was in each document to start with.

\DeclareMathOperator{\proj}{\mathbf{proj}}
\newcommand{\veci}{{\boldsymbol{\hat{\imath}}}}
\newcommand{\vecj}{{\boldsymbol{\hat{\jmath}}}}
\newcommand{\veck}{{\boldsymbol{\hat{k}}}}
\newcommand{\vecl}{\vec{\boldsymbol{\l}}}
\newcommand{\uvec}[1]{\mathbf{\hat{#1}}}
\newcommand{\utan}{\mathbf{\hat{t}}}
\newcommand{\unormal}{\mathbf{\hat{n}}}
\newcommand{\ubinormal}{\mathbf{\hat{b}}}

\newcommand{\dotp}{\bullet}
\newcommand{\cross}{\boldsymbol\times}
\newcommand{\grad}{\boldsymbol\nabla}
\newcommand{\divergence}{\grad\dotp}
\newcommand{\curl}{\grad\cross}
%\DeclareMathOperator{\divergence}{divergence}
%\DeclareMathOperator{\curl}[1]{\grad\cross #1}
\newcommand{\lto}{\mathop{\longrightarrow\,}\limits}

\renewcommand{\bar}{\overline}

\colorlet{textColor}{black}
\colorlet{background}{white}
\colorlet{penColor}{blue!50!black} % Color of a curve in a plot
\colorlet{penColor2}{red!50!black}% Color of a curve in a plot
\colorlet{penColor3}{red!50!blue} % Color of a curve in a plot
\colorlet{penColor4}{green!50!black} % Color of a curve in a plot
\colorlet{penColor5}{orange!80!black} % Color of a curve in a plot
\colorlet{penColor6}{yellow!70!black} % Color of a curve in a plot
\colorlet{fill1}{penColor!20} % Color of fill in a plot
\colorlet{fill2}{penColor2!20} % Color of fill in a plot
\colorlet{fillp}{fill1} % Color of positive area
\colorlet{filln}{penColor2!20} % Color of negative area
\colorlet{fill3}{penColor3!20} % Fill
\colorlet{fill4}{penColor4!20} % Fill
\colorlet{fill5}{penColor5!20} % Fill
\colorlet{gridColor}{gray!50} % Color of grid in a plot

\newcommand{\surfaceColor}{violet}
\newcommand{\surfaceColorTwo}{redyellow}
\newcommand{\sliceColor}{greenyellow}




\pgfmathdeclarefunction{gauss}{2}{% gives gaussian
  \pgfmathparse{1/(#2*sqrt(2*pi))*exp(-((x-#1)^2)/(2*#2^2))}%
}


%%%%%%%%%%%%%
%% Vectors
%%%%%%%%%%%%%

%% Simple horiz vectors
\renewcommand{\vector}[1]{\left\langle #1\right\rangle}


%% %% Complex Horiz Vectors with angle brackets
%% \makeatletter
%% \renewcommand{\vector}[2][ , ]{\left\langle%
%%   \def\nextitem{\def\nextitem{#1}}%
%%   \@for \el:=#2\do{\nextitem\el}\right\rangle%
%% }
%% \makeatother

%% %% Vertical Vectors
%% \def\vector#1{\begin{bmatrix}\vecListA#1,,\end{bmatrix}}
%% \def\vecListA#1,{\if,#1,\else #1\cr \expandafter \vecListA \fi}

%%%%%%%%%%%%%
%% End of vectors
%%%%%%%%%%%%%

%\newcommand{\fullwidth}{}
%\newcommand{\normalwidth}{}



%% makes a snazzy t-chart for evaluating functions
%\newenvironment{tchart}{\rowcolors{2}{}{background!90!textColor}\array}{\endarray}

%%This is to help with formatting on future title pages.
\newenvironment{sectionOutcomes}{}{}



%% Flowchart stuff
%\tikzstyle{startstop} = [rectangle, rounded corners, minimum width=3cm, minimum height=1cm,text centered, draw=black]
%\tikzstyle{question} = [rectangle, minimum width=3cm, minimum height=1cm, text centered, draw=black]
%\tikzstyle{decision} = [trapezium, trapezium left angle=70, trapezium right angle=110, minimum width=3cm, minimum height=1cm, text centered, draw=black]
%\tikzstyle{question} = [rectangle, rounded corners, minimum width=3cm, minimum height=1cm,text centered, draw=black]
%\tikzstyle{process} = [rectangle, minimum width=3cm, minimum height=1cm, text centered, draw=black]
%\tikzstyle{decision} = [trapezium, trapezium left angle=70, trapezium right angle=110, minimum width=3cm, minimum height=1cm, text centered, draw=black]


 
\title{Power} 
\author{Milica Markovic} 
\outcome{Explain the complex power and its parts. Calculate real average power delivered from the generator.}
\begin{document}  
\begin{abstract}  

\end{abstract}  
\maketitle    



The current and voltage phasor transformation is defined as

\begin{eqnarray}
v(t)=\Re\{|V|e^{j\theta_v} e^{j \omega t}\} \\
i(t)=\Re\{|I|e^{j\theta_i} e^{j \omega t}\}
\end{eqnarray}

Where $|V|e^{j\theta_v}$ and $|I|e^{j\theta_i}$ are phasors of voltage and current, and usually denoted with a tilde $\tilde{}$ over capital letters  $\tilde{V}$, $\tilde{I}$.


\begin{eqnarray}
v(t)=\Re\{\tilde{V} e^{j \omega t}\} \\
i(t)=\Re\{\tilde{I} e^{j \omega t}\}
\end{eqnarray}


The real part of a complex number can also be found as $\Re\{z\}=\frac{1}{2}(z +z^*)$, so the above two equations can be re-written as

\begin{eqnarray}
v(t)=\frac{1}{2} ( \tilde{V}e^{j \omega t} + \tilde{V}^*e^{-j \omega t} ) \label{eq:phasorV}\\
i(t)=\frac{1}{2} ( \tilde{I}e^{j \omega t} + \tilde{I}^*e^{-j \omega t} ) \label{eq:phasorI} 
\end{eqnarray}

Power is defined as a product of voltage and current.

\begin{equation}
p(t) = v(t) i(t) \label{eq:power}
\end{equation}

If we replace voltage and current in Equation \ref{eq:power} in the time domain with Equations \ref{eq:phasorV} and \ref{eq:phasorI} we get



\begin{eqnarray}
p(t) =\frac{1}{4} ( \tilde{V}e^{j \omega t} + \tilde{V}^*e^{-j \omega t} ) ( \tilde{I}e^{j \omega t} + \tilde{I}^*e^{-j \omega t} ) 
\end{eqnarray}

Multiplying the terms above, and rearanging, we get:


\begin{eqnarray}
p(t)=\frac{1}{4} (\tilde{V}\tilde{I}^*+ (\tilde{V}\tilde{I}^*)^*+ \tilde{V}\tilde{I} e^{2j \omega t}+ (\tilde{V}\tilde{I} e^{2j \omega t})^*)
\end{eqnarray}

We can again apply equation $\Re\{z\}=\frac{1}{2}(z +z^*)$ to simplify the above equation to

\begin{eqnarray}
p(t)=\frac{1}{2} (\Re\{\tilde{V}\tilde{I}^*\}+ \Re\{\tilde{V}\tilde{I} e^{2j \omega t }\})
\end{eqnarray}

This can also be re-written as

\begin{eqnarray}
p(t)=\frac{1}{2} \Re\{ |V| e^{j\theta_v} |I| e^{-j\theta_i} \}+ \frac{1}{2} \Re\{|V| e^{j\theta_v} |I| e^{j\theta_i} e^{2j \omega t }\} \\
p(t)=\frac{1}{2} \Re\{ |V| |I| e^{j(\theta_v-\theta_i)} \}+ \frac{1}{2} \Re\{|V| |I|  e^{j(\theta_v+\theta_i)}  e^{2j \omega t }\}
\end{eqnarray}


p(t) above is instantaneous power, $S=\tilde{V}\tilde{I}^*=|V| |I| e^{j(\theta_v-\theta_i)}$ is complex power. Complex power has real and reactive parts S=P+jQ. The first part of the equation represents the average real power P delivered to the load $P=\frac{1}{2}\Re\{\tilde{V}\tilde{I}^*\}$, and the second part represents the fluctuating power. We are usually interested in the average real power P delivered to the load.

To find the real power delivered to the load, one would take the real part of the complex power. If we know that the impedance of the load is $Z=R+jX$, the voltage is $ \tilde{V} = Z \tilde{I}$ and we remember that $\tilde{I} \tilde{I}^* = |I|^2$ then the real power is

\begin{eqnarray}
P=\frac{1}{2}\Re\{ \tilde{V} \tilde{I}^*  \} \\
P=\frac{1}{2}\Re\{ (R+jX) \tilde{I} \tilde{I}^*  \} \\
P=\frac{1}{2}\Re\{ (R+jX) |I|^2  \} \\
P=\frac{1}{2}|I|^2 \Re\{ (R+jX)   \} \\
P=\frac{1}{2}R |I|^2 
\end{eqnarray}




\begin{example}
A transmitter operated at 20MHz, Vg=100V with $50 \Omega$ internal impedance is connected to an antenna load through 6.33m of the line. The line is a lossless $50 \Omega$, $\beta=0.595rad/m$. The antenna impedance at 20MHz measures $Z_L=36+j20 \Omega$. 
\begin{enumerate}
\item What is the electrical length of the line? (answer: length=0.6$\lambda$)
\item How much power is delivered to the line? Hint: Find the input impedance, then find the input power as $P_{ave,in}=\frac{1}{2}R_{in} |I_{in}|^2$
\item What is the time-average power absorbed by $Z_L$. $P_{L}=\frac{1}{2} R_L |I_{L}|^2$
\item If now we match load impedance $Z_l$ to 50 Ohm line, what is the input impedance of the line, and how much average power is delivered to the line and load?
\end{enumerate}
\begin{solution}
\begin{description}
\item[a] Electrical length of the line is $0.6 \lambda$.
\item[b]  The input impedance on the line is $Z_{in}=70.8+j27.1 \Omega$.
The load reflection coefficient is $\Gamma_L=0.27 e^{j112^0}$. The 
input reflection coefficient is $\Gamma_{in}= 0.27 e^{-j320} = 0.27 e^{j40^0}$.
The input current is $I_{in}=0.813 e^{-j12.6^0}$. 
The input power is 23.4W.
\item[c] The current at the load is $I_L=1.14 e^{-j223^0}$. The average power at the load is 23.4W (the same as at the input). This is because we are assuming a lossless line, so all power delivered to the line will be delivered to the load.
\item[d] When the load impedance is $50 \Omega$, the average input power is 25W. When the load is matched to the line and generator, we have the maximum power available from the generator delivered to the load.
\end{description}


 
\end{solution}


\end{example}

\end{document} 

